% This is the LaTeX template file for lecture notes for EE 382C/EE 361C/EE 382N/etc.
% This template is based on the template for Prof. Sinclair's CS 270.

\documentclass[twoside]{article}
\usepackage{graphics}
\usepackage{amsmath}
\usepackage{hyperref}
\hypersetup{
    colorlinks=true,
    linkcolor=blue,
    filecolor=magenta,      
    urlcolor=cyan,
}
\urlstyle{same}
\setlength{\oddsidemargin}{0.25 in}
\setlength{\evensidemargin}{-0.25 in}
\setlength{\topmargin}{-0.6 in}
\setlength{\textwidth}{6.5 in}
\setlength{\textheight}{8.5 in}
\setlength{\headsep}{0.75 in}
\setlength{\parindent}{0 in}
\setlength{\parskip}{0.1 in}

\newcommand{\HRule}{\rule{\linewidth}{0.4mm}}

\title{Iterim Report: Package for Controlling Delivery of Messages}
\author{Samuel Cherinet and Robert Pate }
\date{November 2017}

\begin{document}
\maketitle
\HRule

\section{Goal and Use Case}

"Design and implement a package that allows a programmer to specify and control delivery of messages in a distributed program."

As an application developer, I need different instances of my distributed application to communicate with each other. I want to import a package into my code that lets me do that seamlessly with a few commands.

\section{Literature and Other Solutions}

\href{http://zeromq.wdfiles.com/local--files/intro\%3Aread-the-manual/Middleware\%20Trends\%20and\%20Market\%20Leaders\%202011.pdf}{Middleware Trends and Market Leaders 2011} is a paper by A. Dworak, F. Ehm, W. Sliwinski, and M. Sobczak for CERN that analyzes the options for replacing the aging middleware for their 4000 servers and 80,000 devices. 

\href{http://zeromq.org}{ZeroMQ} is "distributed messaging" with many features such as multiple languages, multiple platforms, multiple transport protocols, various patters, open source, and \href{http://zeromq.org/intro:read-the-manual}{good documentation}. The CERN paper rated it the highest and it's had continual development since then.

\href{http://www.inspirel.com/yami4/}{YAMI4} is a "messaging solution for distributed systems" and positions itself as a \href{http://www.inspirel.com/articles/YAMI4_vs_ZeroMQ.html}{"competitor" to ZeroMQ}.

\href{http://www.amqp.org/}{AMQP} stands for "The Advanced Message Queuing Protocol" which is also a good description of it. It became an \href{https://www.oasis-open.org/news/pr/iso-and-iec-approve-oasis-amqp-advanced-message-queuing-protocol}{ISO standard} in 2014 and has various \href{http://www.amqp.org/about/examples}{implementations} by 3rd parties. 


\section{Functions and Restrictions}

\subsection{Restrictions}

\begin{enumerate}
\item Only supports JAVA
\item Only supports TCP
\item Intended for distributed architecture
\item Only support asynchronous messaging
\end{enumerate}

\subsection{Some Example Functions}

\begin{enumerate}
\item bindTo("string") ip address and port
\item Server s = new Server("ip/port string") // takes optional options
\item s.send("Hello")
\item receive - tell the server what to expect
\end{enumerate}

\section{Questions and Next Steps}


\begin{enumerate}
\item Review AMQP to see if it's feasible to build an implementation in a few weeks
\item Decide what level of message ordering we're going to support
\item Review the ZeroMQ and YAMI4 semantics for ideas
\item Outline all the public objects and methods of the package
\item Construct the minimum functioning package 
\item Test the package by replacing messaging code from HW2
\item Iterate and improve
\end{enumerate}


\end{document}

